
\RequirePackage[ngerman=ngerman-x-latest]{hyphsubst}
\documentclass[
        ngerman,
        paper=a4,
        numbers=noendperiod,
]{scrreprt}
\setcounter{secnumdepth}{3}
\setcounter{tocdepth}{3}
% Encoding
\usepackage[utf8]{inputenc}
\usepackage[T1]{fontenc}
% Sprachsupport
\usepackage[ngerman]{babel}
\usepackage{translator}
% Tabellen
\usepackage{booktabs}
\usepackage{tabularx}
\usepackage{pdflscape}
\usepackage{multirow}
% Symbole
\usepackage{eurosym}
% Formeln
\usepackage{amsmath, amsthm, amssymb}
% Formelregeln
\DeclareNewTOC[% 
  counterwithin=chapter, 
  indent=0pt,% kein Einzug im Verzeichnis 
  hang=2em,% Einzug für den Text im Verzeichnis 
  name=equation, 
  type=xequation, 
  nonfloat, 
]{loe} 

\AtBeginDocument{% 
  \newcaptionname{ngerman}\xequationname{Formel}% 
  \newcaptionname{ngerman}\listxequationname{Formelverzeichnis}% 
} 
% Pakete
\usepackage{float}
\usepackage{wrapfig}
\usepackage[babel,german=quotes]{csquotes}
\usepackage[square,sort]{natbib}
\usepackage[hyphens]{url}
\usepackage{setspace}
\onehalfspacing
\usepackage[
        pdftex,
        hyperfigures,
        hyperindex,
        bookmarksnumbered,
        linktoc=all,
        pdfborder={0.25 0.25 0.25},
        %pdfborder={0 0 0},
        pdfpagelayout=TwoColumnRight,
]{hyperref}
\usepackage[all]{hypcap}
\usepackage{lmodern}
\usepackage[final,babel]{microtype}
\usepackage{graphicx}
\usepackage{fancyhdr}
\usepackage[printonlyused]{acronym}

\pagestyle{fancy}
\renewcommand{\chaptermark}[1]{\markboth{#1}{}}
\fancyhf{}
\fancyhead[RE]{\chaptername~\thechapter}
\fancyhead[LO]{\leftmark}
\fancyhead[LE,RO]{\thepage}

%Quellcodes
%Farben
\usepackage{color}
\definecolor{dkgreen}{rgb}{0,0.6,0}
\definecolor{gray}{rgb}{0.5,0.5,0.5}
\definecolor{mauve}{rgb}{0.58,0,0.82}
%Listing einfaerben
\usepackage{listings}
\lstset{numbers=left,
	numberstyle=\tiny,
	numbersep=5pt,
	breaklines=true,
	showstringspaces=false,
	frame=l ,
	xleftmargin=15pt,
	xrightmargin=15pt,
	basicstyle=\ttfamily\scriptsize,
	stepnumber=1,
	keywordstyle=\color{blue},          % keyword style
  	commentstyle=\color{dkgreen},       % comment style
  	stringstyle=\color{mauve}         % string literal style
}
%Sprache Festelegen
\lstset{language=R}

\begin{document}
\begin{titlepage}
    \begin{center}
    \huge \textbf{\textsf{Evaluierung von Paradigmen des Question Answering}} \\
    \vspace{1cm}
    \LARGE\textbf{\textsc{Projektarbeit }}\\
    \vspace{1cm}
    \normalsize
    vorgelegt am: \today \\
    \vspace{2.5cm}
    \large \textbf{Fakultät IV - 
Institut für Wissensbasierte
Systeme und Wissensmanagement, Universität Siegen
}
\linebreak
\linebreak
\begin{figure}[H]
    \centering\includegraphics[width=0.4\linewidth]{images/imageuni.pdf}
    \label{fig:Unilabel}
\end{figure}
    \end{center}
    \vspace{3cm}
    \begin{center}
 \normalsize{
    \begin{tabular}{ll}
    	Eingereicht von: & {Ugur Tigu} \\
    	Studiengang: & Wirtschaftsinformatik, Master of Science (M.Sc.)\\
	Erstprüfer: & Prof. Dr.-Ing. Madjid Fathi \\
	Betreuer: &   Johannes Zenkert\\
    \end{tabular}\\
    }
\end{center}
\end{titlepage}
\setcounter{page}{0}
\pagenumbering{Roman}
\tableofcontents
\clearpage 
\addcontentsline{toc}{chapter}{Abbildungsverzeichnis}
\listoffigures
\clearpage 
\addcontentsline{toc}{chapter}{Tabellenverzeichnis}
\listoftables
\clearpage 
% Kapiteldefinition ohne Nummerierung
\chapter*{Abkürzungsverzeichnis}
 % Abkürzungsverzeichnis soll im Inhaltsverzeichnis erscheinen
\addcontentsline{toc}{chapter}{Abkürzungsverzeichnis} 
\begin{acronym}
% Format der Abkürzungsdefinition: \acro{}[]{}
% {Verweis}[Abkürzung]{ausgeschriebene Abkürzung}

\acro{nlp}[NLP]{Natural Language Processing}
\acro{qa}[QA]{Question Answering}
\acro{ir}[IR]{Information Retrieval}
\acro{pos}[POS]{Part-of-speech}

\end{acronym}
\clearpage 
\addcontentsline{toc}{chapter}{Formelverzeichnis} 
\listofxequations
\clearpage
\setcounter{page}{1}
\pagenumbering{arabic}





\chapter{Einleitung}
\section{Motivation}
\section{Hauptidee der Projektarbeit}
\section{Verwandte Arbeiten}









\chapter{Question Answering}
\section{Die Geschihte des QA}
\section{Allgemeine Systemarchitektur}
\section{Fragetypen und Antworttypen}
\section{Evaluationsmethoden des QA}
\section{Stand der Forschung}






























\chapter{Theoretische Grundlagen des QA}
\section{Natural Language Processing}
\section{Machine Learning}
\chapter{Paradigmen des QA}
\section{IR-basiert}
\section{NLP-basiert}
\section{Wissensbasiert}
\section{Hybrid}
\chapter{Analyse der Paradigmen des QA}
\section{Techniken, Algorithmen, Frameworks und Tools}
\section{Einsatz der Paradigmen}
\section{Analyse der Evaluationsmethoden}
\chapter{Fazit}
\appendix 
\chapter{Anhang}
\label{chapter:Anhang}%


\clearpage
        \phantomsection % damit das pdf bookmark an die richtige Stelle zeigt
        \pdfbookmark{Literaturverzeichnis}{bibliography}
        
        % zeigt immer alle definierten Quellen an, auch wenn diese nicht verwendet werden
        %\nocite{*}
        \bibliographystyle{apalike}
        \addcontentsline{toc}{chapter}{Literaturverzeichnis}
        \bibliography{literatur}




\chapter*{Erklärung}
Hiermit versichere ich, dass ich die vorliegende Arbeit selbstständig verfasst und keine anderen als die angegebenen Quellen und Hilfsmittel benutzt habe, insbesondere keine anderen als die angegebenen Informationen aus dem Internet. Diejenigen Paragraphen der für mich gültigen Prüfungsordnung, welche etwaige Betrugsversuche betreffen, habe ich zur Kenntnis genommen. Der Speicherung meiner Master-Arbeit zum Zweck der Plagiatsprüfung stimme ich zu. Ich versichere, dass die elektronische Version mit der gedruckten Version inhaltlich übereinstimmt.\newline
\linebreak
\linebreak
\linebreak
Bielefeld, den 01.04.2020\newline
(Ort) (Datum)\newline
\linebreak
\linebreak
\linebreak
..................................\newline
(Unterschrift)
\end{document}