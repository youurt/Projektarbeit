\documentclass[../Masterarbeit.tex]{subfiles}
\begin{document}
\enquote{Wolfram|Alpha is not a search engine.} \citep{WolframAlpha} Weil die Menschen mehr als eine Suchmaschine gebraucht haben und immer mehr dem Computer Fragen stellten entstanden zunächst Projekte wie Wolfram|Alpha entstanden. Herkömmliche Suchmaschinen arbeiten unter der Annahme, dass die bestimmte Webseiten wonach der Nutzer gesucht hat, die gewünschten Informationen enthalten. Anders bei Wolfram|Alpha wo durch die Suche. Wolfram|Alpha bezeichnet sich selbst als \enquote{Wissensmaschine}. Sie generiert die Antwort, indem sie mit der eigenen internen Wissensbasis arbeitet und nicht nur einen Index wiedergibt. Als zunächst Wolfram|Alpha veröffentlicht wurde, war der Hype sehr groß. Es sollte den von Google dominierten Markt für Suchmaschinen revolutionieren, doch Wolfram|Alpha ist kein Ersatz für herkömmliche Suchmaschinen.

2011 gelang es IBM Watson das Jeopardy! Fernsehprgramm zu gewinnen \citep{Markoff2011ComputerNot}. IBM musst mit Watson gegen Menschen antreten und in natürlicher Sprache gestellte Quizfragen beantworten. Der Computer hatte dabei schnell und prezise aus einer sehr großen Anzahl von Antwortmöglichkeiten, die richtige Antwort auszuwählen. Dieses war eine Demonstration von IBM, wie fortgeschritten die Technik angekommen ist.

Die Entwicklung im Bereich QA ging weiter und auch Google hatte in diesem Bereich eine neue Möglichkeit entdeckt, das QA in die herrkömmliche Suche zu integrieren. Im Mai 2012 hat Google das sogenannte \enquote{Knowledge Graph} eingeführt. Ein Wissensgraph, das reale Entitäten und ihre Beziehungen zueinander verknüpft. Zu den Objekten gehören z.B.: Sehenswürdigkeiten, Prominente, Städte, Sportmannschaften, Gebäude, Filme, Himmelsobjekte, Kunstwerke und mehr. Das Diagramm verbessert die Google-Suche auf drei Arten: die Auflösung von Mehrdeutigkeiten bei Suchanfragen, die Zusammenfassung der wichtigsten Fakten und die explorative Suche \citep{SteinerAddingGraph}.


\end{document}