\documentclass[../Masterarbeit.tex]{subfiles}
\begin{document}
Erste Adaptionen von QA Systemen konnten nur mit strukturierten Listen oder Teilen von begrenzten Datenbanksystemen umgehen. Datenbanken über eine bestimmte Domain wurden entwickelt speziell eine Aufgabe zu lösen. Im Prinzip wurden Datenbanken textbasiert oder durch logische Inferenzen zu Wissensdatenbanken zusammengefasst. Die Antwort wurde dabei gefunden, indem das System eine Übereinstimmung im Textabschnitt gefunden hatte oder durch Fakten das Problem lösen konnte.

Mit \enquote{BASEBALL}  \citep{Green1961Baseball:Question-answerer} wurden Baseball-Spiele analysiert.  Auf die Frage \enquote{Wie haben die Yankees im Juli gespielt?} konnte das QA-System eine Antwort geben. Das QA-System hatte also eine beschränkte, geschlossene Domain. Zunächst wurde eine syntaktische Analyse der Frage ausgeführt. Die Frage wurde dabei in seine einzelnen Elemente zerlegt und es entstanden Teile wie \enquote{Wie?}, \enquote{die Yankees} und \enquote{in Juli}. Da es nur Baseball-Spiele ging, konnte das QA-System direkt mit der semantischen Analyse beginnen. Die semantische Analyse verknüpfte dabei, \enquote{die Yankees} mit der Kondition \enquote{Wie?}. In diesem Fall ist das \enquote{Wie?} die Frage nach dem \enquote{Wie gespielt?}.

Es folgten weitere Anwendungen, wo Enzyklopädien benutzt wurden, um Fragen zu beantworten. Mit \enquote{Protosynthex}\citep{Simmons1964IndexingQuestions} entstand eines der ersten wirklichen QA-Systeme. Um eine Frage zu beantworten, musste diese Systeme zunächst eine Abfrage ausführen. Ähnlich wie bei einer Datenbankabfrage entstand aus den einzelnen Elementen der Frage eine strukturierte Abfrage. Das System gab auf diese strukturierte Abfrage dann bestimmte Antwort Kandidaten. Diese Kandidaten waren die möglichen Antworten. Auf die Frage wurden dann in eine Reihenfolge gebracht. Je ähnlicher die Frage mit der Antwort war, desto höher war der Rang des Kandidaten. Auf die Frage \enquote{Was essen Würmer?} gab das System die korrekte Antwort \enquote{Würmer essen Gras.}, da sowohl die Frage, als auch die Antwort die Abhängigkeit \enquote{essen} beinhalteten. 

Ein weiterer wichtiger Abschnitt bei den ersten QA-Systemen war \enquote{LUNAR} \citep{Woods1978SemanticsAnswering}. Dieses QA-System war eine Brücke zwischen Menschen, die mit natürlicher Sprache kommunizieren und der Chemie, die strukturierte Fakten abbildet. LUNAR beantwortete  Fragen wie \enquote{Gibt es Proben mit 13 Prozent Aluminium?}.


Die ersten QA-Systeme waren im Grunde wissensbasierte Systeme. Mit dem Fortschritt des Internets kamen aber immer mehr IR-basierte QA-Systeme in den Vordergrund und erst später kam die Kombination dieser beiden Paradigmen, die hybriden Systeme, welche sich noch in dem Beginn ihrer Entwicklung befinden. 

Aus den ersten Implementierungen der QA-Systeme können wir aber lernen, indem wir nicht dieselben Fehler begehen wie diese Systeme. In den 70er Jahren gab es noch sehr wenig Daten, und QA-Systeme leben von Daten. Den wirklichen Durchbruch haben QA-Systeme erst Mitte der 2000'er erhalten. Große Mengen an Daten und ML-Algorithmen, welche diese Daten umwandeln und beherrschen sind der Grund. Es ist nicht ausreichend, nur eine Quelle zu haben, um die Komplexität zu beherrschen. Ein Wort kann in der natürlichen Sprache mehrere Bedeutungen haben. Außerdem ist eine reine Regelbasierte Behandlung der Frage nicht ausreichend.
\end{document}